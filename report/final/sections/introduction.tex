The diversity that appears on planet Earth is broad but reuse is a common trait among all animals. Each face is 
complex in its own right, but what is the ultimate limit in possible facial configurations? Each facial structure
is evolved to attract mates and/or optimize placement of sensors to increase survivability. The possible configurations
of faces though is limited by multiple factors such as weight, surface area, etc. these factors place an upper limit
on the possible faces that can be seen on the planet Earth. These optimization factors combined with the non-random 
starting point of currently available facial structures converts an immeasurable problem to one of surprisingly low
dimensionality \cite{sirovich1987low}. This paper will focus on human and human-like faces for its distortions.

Human like faces are interesting in that they demonstrate the property of repeatability with variation \cite{stanley2007compositional}.
Most animal faces exhibit these same qualities as well, though seeing a warped animal face may not seem as interesting to
a human observer compared with a distortion of a human face since the incongruities of distorted faces in members of his or her own
species seem a bit more disturbing. This process due potentially to facial recognition circuits buried deep within the human brain that have evolved
for quick unsupervised feature detection and identification in conjunction with other circuitry designed to recognize specifically members of
the human race. This understandability of the interestingness of evolved faces directly ties to the ability to quickly verify
results obtained by the algorithmic process and is a cornerstone to this paper's research.

The overall goal of this paper is to produce new plausible facial structures and explore how these facial structures evolve. Though,
as mentioned earlier, the dimensionality for human faces is bounded, this bound is unexplored. Taking this into consideration
this paper will utilize a evolutionary process to search for new and novel faces. There are multiple search algorithms
in practice but most of them require some form of objective function and it is currently unknown how to objectify interesting faces.
Taking this into account this paper will utilize a user guided search where the individual determines what is interesting. This, though,
is not enough as the user may not know what is interesting until it is presented. The algorithm could be a random search but due to the
size of the search space it may require an unrealistically long duration before interesting results are uncovered. 

To solve these issues the algorithm used is NeuroEvolution and Augmenting Topologies \cite{stanley2002evolving} hereby referred to as NEAT.
NEAT allows for directed evolution of neural networks towards a goal using the concepts of speciation (evolving new species), historical markings,
and directed complexification. The NEAT process is utilized in conjunction with Compositional Pattern Producing Networks \cite{stanley2007compositional}
(CPPNs) and interactive evolution in order to evolve distortions more desirable to the human operator. This paper will seek to provide evidence that not
only can interesting images be found by humans but that these can be found with great efficiency and that, in the process of exploration, a
deeper process starts to unveil itself.

This paper will provide evidence that evolution of images needs not only a directed search but also an understanding of the domain
in order to correctly capture interesting facial structures. Exploration of the process and results of the evolution will take place
in this paper including examination of the interesting results gleaned. These results hint at some deeper understanding of not only what
humans find interesting but how human interaction changes the exploration and exploitation paradigm. This paper along with these
theoretical underpinnings explores an evolutionary architecture that produces more varied results and allows for faster exploitation
of those figures and features found in the exploration process.