The diversity that appears on planet Earth is broad but reuse is a common trait among all animals. Each face is 
complex in its own right, but what is the ultimate limit in possible facial configurations? Each facial structure
is evolved to attract mates and/or optimize placement of sensors to increase survivability. The possible configurations
of faces though is limited by multiple factors such as weight, surface area, etc. these factors place an upper limit
on the possible faces that can be seen on the planet Earth. These optimization factors combined with the non-random 
starting point of currently available facial structures converts an immeasurable problem to one of surprisingly low
dimensionality \cite{sirovich1987low}.This paper will limit the discussion to Human and Human-like faces in order to constrain the
produce empirically verifiable results.

Human like faces are interesting in that they represent the property of repeatability with variation \cite{stanley2007compositional}.
Although most animal faces do carry these same attributes, its inherently easier for a Human to identify and understand facial 
features and structures. This process is mostly due to facial recognition circuits buried deep within the Human brain that have evolved
for quick unsupervised feature detection and identification of fellow Humans. This understandability of the interestingness of evolved 
faces directly ties to the ability to quickly verify results obtained by the algorithmic process and is a cornerstone to this paper's research.

The overall goal of this paper is to produce new plausible facial structures and explore how these facial structures evolve.Although,
as mentioned earlier the dimensionality for human faces is bounded, but this bound is unexplored. Taking this into consideration
this paper will utilize a searching algorithm in order to search for new and novel faces. There are multiple searching algorithms
in practice, but most of them require some form of objective function, and it is currently unknown how to objectify interesting faces.
Taking this into account this paper will utilize a user guided search where the individual determines what is interesting. This though
is not enough as the user may not know what is interesting until it is presented. The algorithm could be a random search, but due to the
size of the search space it may take a user years of clicking in order to find interesting results. 

To solve these issues the algorithm that was used is Neuroevolution and Augmenting Topologies \cite{stanley2002evolving} hereby referred to as NEAT.
NEAT allows for directed evolution of neural networks towards a goal using the concepts of speciation (evolving new species), historical markings,
and directed complexification. NEAT utilized with Compositional Pattern Producing Neural Networks \cite{stanley2007compositional} and interactive evolution
in order to produce more lifelike and non-objective driven images. This paper will seek to provide evidence that not only can interesting images be found
by Humans but these can be found with great efficiency, and that in the process of exploration a deeper process starts to unveil itself. 

This paper will provide evidence that evolution of images need not only a directed search but an understanding of the domain in order to correctly capture
interesting facial structures. Exploration of the process and results of the evolution will take place in this paper including examination of interesting
results gleamed from this exploration process. These interesting results hint at some deeper understanding of not only what Humans find interesting but how Human 
interaction changes the exploration/exploitation paradigm. This paper along with these theoretical underpinnings explores a new image evolution architecture
that produces more varied results and allows for faster exploitation of those figures and features found in the exploration process.