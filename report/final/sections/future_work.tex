The experiments presented in this report have yielded a fairly diverse and interesting set of outcomes but by no means is it argued that the results and the experimental method is optimal, assuming there is an optimum to be found. From this point there are many avenues of research that might yield more varied, structured, and interesting results. Here some ideas for future progress will be presented.

Given the technology as it stands it would be interesting to see more results of a more widespread use of the application. Due to the inherent biases in the operators conducting the research, some artificial limits due to operator likes and dislikes could certainly be constraining the discovery of many interesting distortions. A PicBreeder style community approach might be interesting to see.

Two methods were developed as a result of this research to boost the generation of complexity in the networks: the more conservative cross correlation approach and the bit more extreme uncorrelated random burst approach. The random burst approach is not necessarily well suited for building desirable complexity as opposed to simply random complexity, but it is very fast. The cross correlation does a somewhat better job, but it can only detect pixel by pixel image similarities, and only among the set the cross correlation is run against. Because of the time required to run the algorithm the only images correlated were among a single generation. This removes the possibility of ignoring similar images found in past generations. These two limitations, that of speed and that of limited history, can perhaps be mitigated by uncovering better approaches to the calculations or the method used entirely. Developing filtering and complexification methodologies that are not quite so time consuming might make the process more useful and friendly to an operator with limited time.

As mentioned previously, though some ideas were explored this research was unable to uncover an effective method for creating pose and proportion invariant distortions. It is believed that for the distortions to have any sort of practical application they should be less dependent on the unique spatial properties and orientations of the images they're evolved with. It would certainly be interesting to see this hurdle overcome.

This research has unveiled some interesting possibilities of the technique when used on cartoon images. As mentioned previously this technique could be useful in artificial user-created worlds. It is feasible to see how an algorithm might be able to convert the deformed cartoon image into a more structured creature. It would be interesting to see this avenue explored further.

This has been a fairly limited selection of the possibilities for expansion and it is easy to see this technology is quite open ended in method and application so the potential for future work is necessarily unbounded as far as one can see.